%%%%%%%%%%%%%%%%%%%%%%%%%%%%%%%%%%%%%%%%%%%%%%%%%%%%%%%%%%%%%%%%%%%%%%%%%%%%%%%%%%%%%%%%
 %%%                               OXFORD THESIS TEMPLATE                            %%%
%
% Use this template to produce a standard thesis that meets the Oxford University
% requirements for DPhil submission
%
% Originally by Keith A. Gillow (gillow@maths.ox.ac.uk), 1997
% Modified by Sam Evans (sam@samuelevansresearch.org), 2007
% Modified by John McManigle (john@oxfordechoes.com), 2015
% Modified by Mónica Sagastuy-Breña (m.sgstyb@gmail.com), 2024
%
% This version Copyright (c) 2024 Mónica Sagastuy-Breña
%
% Broad permissions are granted to use, modify, and distribute this software
% as specified in the MIT License included in this distribution's LICENSE file.
%%%%%%%%%%%%%%%%%%%%%%%%%%%%%%%%%%%%%%%%%%%%%%%%%%%%%%%%%%%%%%%%%%%%%%%%%%%%%%%%%%%%%%%%


%%%%% CHOOSE PAGE LAYOUT
% This one will format for two-sided binding (ie left and right pages have mirror margins; blank pages inserted where needed):
%\documentclass[a4paper,twoside]{ociamthesis}

% This one will format for PDF output (ie equal margins, no extra blank pages):
\documentclass[a4paper,nobind,twoside]{ociamthesis} 


%%%%% SELECT YOUR DRAFT OPTIONS
% Three options going on here; use in any combination.  But remember to turn the first two off before generating a PDF to send to the printer!

% This adds a "DRAFT" footer to every normal page.  (The first page of each chapter is not a "normal" page.)
%\fancyfoot[C]{\emph{DRAFT}}  

% This highlights (in blue) corrections marked with (for words) \mccorrect{blah} or (for whole
% paragraphs) \begin{mccorrection} . . . \end{mccorrection}.  This can be useful for sending a PDF of
% your corrected thesis to your examiners for review.  Turn it off, and the blue disappears.
%\correctionstrue


%%%%% BIBLIOGRAPHY SETUP
% Note that your bibliography will require some tweaking depending on your department, preferred format, etc.
% The options included below are just very basic "sciencey" and "humanitiesey" options to get started.
% If you've not used LaTeX before, I recommend reading a little about biblatex/biber and getting started with it.
% If you're already a LaTeX pro and are used to natbib or something, modify as necessary.
% Either way, you'll have to choose and configure an appropriate bibliography format...

% The science-type option: numerical in-text citation with references in order of appearance.
%\usepackage[style=ieee, sorting=none, backend=biber, doi=false, isbn=false]{biblatex}
\usepackage{natbib}
\AtBeginDocument{\renewcommand{\bibname}{References}}

% Please add the following required packages to your document preamble:
\usepackage[justification=centering, labelfont=bf]{caption}
\usepackage{multirow}
\usepackage{graphicx}
\usepackage{xcolor}
\usepackage{float}

\usepackage{amsmath}
\usepackage{graphicx}
\usepackage{titlesec}
\usepackage{setspace}
\usepackage{fancyhdr}
\usepackage{caption}
\usepackage{subcaption}
\usepackage{etoolbox}
\usepackage{acro}
\usepackage[bottom]{footmisc}
\usepackage{fancyhdr}
%\usepackage{siunitx}
\usepackage{textgreek}
\usepackage{pgf-pie}
\usepackage{chemist}
\usepackage{physics}
\usepackage{tikz}
\usetikzlibrary{shapes.geometric, arrows}
\usetikzlibrary{shapes,arrows,positioning}
\usetikzlibrary{positioning}
\usepackage{listings}
\usetikzlibrary{calc,trees,positioning,arrows,chains,shapes.geometric,%
decorations.pathreplacing,decorations.pathmorphing,shapes,%
matrix,shapes.symbols,plotmarks,decorations.markings,shadows}
\usepackage{booktabs}
\usepackage{tabu}
\usepackage{indentfirst}


\usepackage{lipsum}

% MSB: To circumvent warning of \degree already defined in gensymb.sty and redefined in ociamthesis.cls to define unabbreviated degree title for titlepage without having to edit the .sty file
\let\olddegree\degree  % Store \degree in \olddegree
\let\degree\relax      % Remove definition of \degree
\usepackage{gensymb}   % This creates \degree
\let\degreesymb\degree % Store new \degree in \degreesymb
\let\degree\olddegree  % Restore \degree to its old definition \olddegree
%%

\usepackage{hyperref}
\hypersetup{
    colorlinks=true,
    linkcolor=indigoDye,
    filecolor=magenta,      
    urlcolor=teal,
    citecolor=oxBlue,
    pdftitle={Oxford-DPhil-Thesis-MSB},
    }

% Uncomment this if you want equation numbers per section (2.3.12), instead of per chapter (2.18):
%\numberwithin{equation}{subsection}


%%%%% THESIS / TITLE PAGE INFORMATION
% Everybody needs to complete the following:
\title{Your DPhil Title}
\author{Your Name}
\college{Your College}
\department{Your Department}
% Your full degree name.  (But remember that DPhils aren't "in" anything.  They're just DPhils.)
\degree{Doctor of Philosophy}
% Term and year of submission, or date if your board requires (eg most masters)
\degreedate{Trinity 2025}


%%%%% YOUR OWN PERSONAL MACROS
% This is a good place to dump your own LaTeX macros as they come up.

% To make text superscripts shortcuts
\renewcommand{\th}{\textsuperscript{th}} % ex: I won 4\th place
\newcommand{\nd}{\textsuperscript{nd}}
\renewcommand{\st}{\textsuperscript{st}}
\newcommand{\rd}{\textsuperscript{rd}}

% MSB: Defined colors for hypersetup and the thesis in general
\definecolor{gray}{RGB}{145, 145, 145}
\definecolor{pewter}{RGB}{115, 171, 182}
\definecolor{teal}{RGB}{79, 125, 145}
\definecolor{indigoDye}{RGB}{40, 79, 108}
\definecolor{oxBlue}{RGB}{0, 33, 71}



%%%%%%%%%%%%%%%%%%%%%%%%%%%%%%%%%%%%%%%%%%%%%%%%%%%%%%%%%%%%%%%%%%%%%%%%%%%%%%%%%%
%%%%%                    THE ACTUAL DOCUMENT STARTS HERE                     %%%%%
%%%%%%%%%%%%%%%%%%%%%%%%%%%%%%%%%%%%%%%%%%%%%%%%%%%%%%%%%%%%%%%%%%%%%%%%%%%%%%%%%%

\begin{document}

%%%%% CHOOSE YOUR LINE SPACING HERE
% This is the official option.  Use it for your submission copy and library copy:
\setlength{\textbaselineskip}{22pt plus2pt}
% This is closer spacing (about 1.5-spaced) that you might prefer for your personal copies:
%\setlength{\textbaselineskip}{18pt plus2pt minus1pt}

% You can set the spacing here for the roman-numbered pages (acknowledgements, table of contents, etc.)
\setlength{\frontmatterbaselineskip}{17pt plus1pt minus1pt}

% Leave this line alone; it gets things started for the real document.
\setlength{\baselineskip}{\textbaselineskip}

% MSB: define vertical separation from header for all pages, fixes large space at top of page in pages after chapter titlepage
\setlength{\headsep}{5mm}

%%%%% CHOOSE YOUR SECTION NUMBERING DEPTH HERE
% You have two choices.  First, how far down are sections numbered?  (Below that, they're named but
% don't get numbers.)  Second, what level of section appears in the table of contents?  These don't have
% to match: you can have numbered sections that don't show up in the ToC, or unnumbered sections that
% do.  Throughout, 0 = chapter; 1 = section; 2 = subsection; 3 = subsubsection, 4 = paragraph...

% The level that gets a number:
\setcounter{secnumdepth}{2}
% The level that shows up in the ToC:
\setcounter{tocdepth}{1}


%%%%% ABSTRACT SEPARATE
% This is used to create the separate, one-page abstract that you are required to hand into the Exam
% Schools.  You can comment it out to generate a PDF for printing or whatnot.
%\begin{abstractseparate}
	%%\vspace{20pt}

Input Abstract Text % Create an abstract.tex file in the 'text' folder for your abstract.
%\end{abstractseparate}


% JEM: Pages are roman numbered from here, though page numbers are invisible until ToC.  This is in
% keeping with most typesetting conventions.
\begin{romanpages}

% Title page is created here
\maketitle

% Include a tickbox to certify this is your own work unless stated otherwise
%\originalitytext


%%%%% DEDICATION -- If you'd like one, un-comment the following.
%\begin{dedication}
%This thesis is dedicated to\\
%someone\\
%for some special reason\\
%\end{dedication}

%%%%% ACKNOWLEDGEMENTS -- Nothing to do here except comment out if you don't want it.
\begin{acknowledgements}
 	%\vspace{20pt}

Acknowledgements
\end{acknowledgements}

%%%%% ABSTRACT -- Nothing to do here except comment out if you don't want it.
\begin{abstract}
	%\vspace{20pt}

Input Abstract Text
\end{abstract}

%%%%%%%%%%%%%%%%%%%%%%%%%%%%%%%%%%%%%%%%%%%%%%%%%%%%%%%%%%
%%%     MINI TABLES OF CONTENTS, FIGURES, TABLES       %%%
%%%%%%%%%%%%%%%%%%%%%%%%%%%%%%%%%%%%%%%%%%%%%%%%%%%%%%%%%%
% This lays the groundwork for per-chapter, mini tables of contents.  Comment the following line
% (and remove \minitoc from the chapter files) if you don't want this.  Un-comment either of the
% next two lines if you want a per-chapter list of figures or tables.

\dominitoc % include a mini table of contents
%\dominilof  % include a mini list of figures
%\dominilot  % include a mini list of tables

% This aligns the bottom of the text of each page.  It generally makes things look better.
\flushbottom

%%%%%%%%%%%%%%%%%%%%%%%%%%%%%%%%%%%%%%%%%%%%%%%%%%%%%%%%%%
%%%      MAIN TOC, LIST OF TABLES, LIST OF FIGS        %%%
%%%%%%%%%%%%%%%%%%%%%%%%%%%%%%%%%%%%%%%%%%%%%%%%%%%%%%%%%%
% Comment any if not required
\tableofcontents
\listoffigures
\mtcaddchapter   % \mtcaddchapter is needed when adding a non-chapter (but chapter-like) entity to avoid confusing minitoc

\listoftables
\mtcaddchapter

%%%%%%%%%%%%%%%%%%%%%%%%%%%%%%%%%%%%%%%%%%%%%%%%%%%%%%%%%%
%%%              LIST OF ABBREVIATIONS                 %%%
%%%%%%%%%%%%%%%%%%%%%%%%%%%%%%%%%%%%%%%%%%%%%%%%%%%%%%%%%%
% This example includes a list of abbreviations.  Look at text/abbreviations.tex to see how that file is
% formatted.  The template can handle any kind of list though, so this might be a good place for a
% glossary, etc.
%!TEX root = ../Oxford_Thesis.tex
%\vspace{20pt}
%\begin{mclistof}{List Title, e.g. Glossary, or List of Abbreviations}{max Length of bold terms}

\begin{mclistof}{List of Abbreviations}{3.2cm}

\item[LoA] List of Abbreviations



\end{mclistof} 


% The Roman pages, like the Roman Empire, must come to its inevitable close.
\end{romanpages}
\cleardoublepage

%%%%%%%%%%%%%%%%%%%%%%%%%%%%%%%%%%%%%%%%%%%%%%%%%%%%%%%%%%
%%%                       CHAPTERS                     %%%
%%%%%%%%%%%%%%%%%%%%%%%%%%%%%%%%%%%%%%%%%%%%%%%%%%%%%%%%%%

%!TEX root = ../Oxford_Thesis.tex
\chapter{\label{ch1: intro}Introduction} 
%%
\minitoc
%% Suggested outline for the introduction, edit content as required.
This thesis aims to answer the question \emph{main research question}.


\section{Aim and Objectives}


\section{Scope}


\section{Thesis Structure}
%% Example for this section, description of the narrative as brief summary of each chapters' contributions
\noindent This thesis is organised as follows:

\textbf{Chapter \ref{ch1: intro}} introduces the context [...]
\textbf{Chapter \ref{ch2: litreview}} includes an extensive literature review, highlighting the research gaps that are addressed in the following chapters [...]
\textbf{Chapter \ref{ch3: first-core-chapter}} description of contributions of the chapter, and why it is important [...]
\textbf{Chapter \ref{ch4: second-core-chapter}} description of contributions of the chapter, and why it is important [...]
\textbf{Chapter \ref{ch5: third-core-chapter}} description of contributions of the chapter, and why it is important [...]
\textbf{Chapter \ref{ch6: conclusions}} summarizes the contributions of this work and emphasizes how the research question and sub-questions are answered. Future areas of work are described in this section [...]


\chapter{\label{ch2: litreview} Literature Review}
\minitoc
\section{Section 1} 
\label{sec: ch2-section-1}
Sectionize literature review through classification (e.g. Linnaeus)


\section{Gap Analysis}
\label{sec: ch2-Research-gap}
Some initial gaps identified:
\begin{itemize}
    \item Gap 1 
    \item Gap 2
    \item Gap 3
\end{itemize}

\chapter{Chapter Title}
\label{ch3: first-core-chapter}
\minitoc

In this chapter, sub-question 1 "..." is addressed.
%!TEX root = ../Oxford_Thesis.tex
\chapter{Chapter Title}
\label{ch4: second-core-chapter}
\minitoc

In this chapter, sub-question 2 "..." is addressed.
\chapter{Chapter Title}
\label{ch5: third-core-chapter}
\minitoc

In this chapter, sub-question 3 "..." is addressed.

\chapter{Conclusions}
\label{ch6: conclusions}
\minitoc

This thesis addresses the question of \emph{Re-state overall PhD Thesis Research Question}.


\section{Future Work}
Summarize limitations in a concise manner, state potential future work areas.

%%%%%%%%%%%%%%%%%%%%%%%%%%%%%%%%%%%%%%%%%%%%%%%%%%%%%%%%%%
%%%                       APPENDICES                   %%%
%%%%%%%%%%%%%%%%%%%%%%%%%%%%%%%%%%%%%%%%%%%%%%%%%%%%%%%%%%
% Starts lettered appendices, adds a heading in table of contents, and adds a
% page that just says "Appendices" to signal the end of your main text. Use
% \include{} command to add appendices from teh text folder

\startappendices

%!TEX root = ../Oxford_Thesis.tex
\chapter{Appendix}
\label{ch: appendix-1}

\section{This is Appendix 1 section 1}
Include e.g. GitHub Repositories with code, datasets created, extensive tables not in main text, etc.




%%%%%%%%%%%%%%%%%%%%%%%%%%%%%%%%%%%%%%%%%%%%%%%%%%%%%%%%%%
%%%                       REFERENCES                   %%%
%%%%%%%%%%%%%%%%%%%%%%%%%%%%%%%%%%%%%%%%%%%%%%%%%%%%%%%%%%

\clearpage
%\addcontentsline{toc}{chapter}{References}
{\footnotesize\setstretch{1.2}
\bibliography{references}
\bibliographystyle{IEEEtranN}
}

\end{document}
